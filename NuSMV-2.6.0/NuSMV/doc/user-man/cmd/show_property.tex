% -*-latex-*-
\begin{nusmvCommand} {show\_property} {Shows the currently stored properties}

\cmdLine{show\_property [-h] [-n idx | -P "name"] [-c | -l | -i | -s | -q] [-f | -v | -u] [-m | -o] [-F format]}

Shows the properties currently stored in the list of properties. This
list is initialized with the properties (CTL, LTL, INVAR, COMPUTE)
present in the input file, if any; then all of the properties added by
the user with the relative \command{check\_property} or \command{add\_property}
commands are appended to this list.  For every property, the following
informations are displayed:
\begin{itemize}
\item the identifier of the property (a progressive number);
\item the property name if available; 
\item the property formula; 
\item the type (CTL, LTL, INVAR, PSL, COMPUTE)
\item the status of the formula (Unchecked, True, False) or the result
  of the quantitative expression, if any (it can be infinite);
\item if the formula has been found to be false, the index number of
  the corresponding counterexample trace.
\end{itemize}
By default, all the properties currently stored in the list of
properties are shown. Specifying the suitable options, properties with
a certain status (Unchecked, True, False) and/or of a certain type
(e.g. CTL, LTL), or with a given identifier, it is possible to let the
system show a restricted set of properties. It is allowed to insert
only one option per status and one option per type.

\begin{cmdOpt}
\opt{-P \parameter{name}}{Prints out the property named "name"}
\opt{-n \parameter{idx}}{Prints out the property numbered "idx"}
\opt{-c}{Prints only CTL properties}
\opt{-l}{Prints only LTL properties}
\opt{-i}{Prints only INVAR properties}
\opt{-q}{Prints only COMPUTE properties}
\opt{-u}{Prints only unchecked properties}
\opt{-t}{Prints only those properties found to be true}
\opt{-f}{Prints only those properties found to be false}
\opt{-s}{Prints the number of stored properties}
\opt{-o \parameter{\filename{filename}}} {Writes the output generated
  by the command to \filename{filename}}
\opt{-F \parameter{\filename{format}}} {Prints with the specified
  format.  {\it tabular} and {\it xml} are common formats, however use
  \command{-F help} to see all available formats.}
\opt{-m}{Pipes the output through the program specified by the PAGER
  shell variable if defined, else through the UNIX "more" command}
\end{cmdOpt}

\end{nusmvCommand}
